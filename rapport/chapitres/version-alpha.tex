\section{Backlog/roadmap du sprint Alpha}

L’objectif de ce sprint est de poser les fondations et délivrer les fonctionnalités de bases attendues par l’utilisateur, 
c'est-à- dire la compréhension des dispositions possibles et impossibles.

Pour ce faire le backlog suivant a été "embarqué" dans cette version et a résulté en la roadmap suivante:

%TODO insérer roadmap

\section{Tests}


\noindent%
\begin{adjustwidth}{-1.5cm}{0cm}

    \renewcommand{\arraystretch}{1.2}
    {\setlength{\tabcolsep}{1.5 mm}
        \begin{testtabular}{|m{0.6cm}|m{5.5cm}|m{8cm}|m{2cm}|c|} \hline
            id                                                                             & Sujet                                                                                & Test d'acceptance                                                                                        & Méthode de test & Résultat \\ \hline
            518                                                                            & TACHE Utiliser la fonction pour renvoyer la réponse au gestionnaire de dojo          &
            Quand le programme est lancé et que l'utilisateur saisi comme attendu,
            une réponse lui est renvoyée dans la console                                   & Manuel                                                                               & OK                                                                                                                                    \\ \hline
            517                                                                            & TACHE Créer une fonction permettant de calculer le nombre de tatamis nécessaires
            étant donné des dimensions de dojo et qu'une solution existe                   &
            Quand la fonction est exécutée, elle retourne le nombre de tatamis nécessaires &
            Automatisé                                                                     & OK                                                                                                                                                                                                                           \\ \hline
            \multirow{3}{0.6cm}{516}                                                       & \multirow{3}{5.5cm}{TACHE Permettre au gestionnaire de dojo de poser cette question} & 1. Quand le programme est lancé, il demande a l'utilisateur la largeur du dojo                           & Automatisé      & OK       \\ \cline{3-5}
                                                                                           &                                                                                      & 2. Quand le programme est lancé, il demande a l'utilisateur la longueur du dojo                          & Automatisé      & OK       \\ \cline{3-5}
                                                                                           &                                                                                      & 3. Quand le programme est lancé, une option est disponible pour l'utilisateur pour poser cette question. & Automatisé      & OK       \\ \hline
        \end{testtabular}}
\end{adjustwidth}


\noindent%
\begin{adjustwidth}{-1.5cm}{0cm}

    \renewcommand{\arraystretch}{1.2}
    {\setlength{\tabcolsep}{1.5 mm}
        \begin{testtabular}{|m{0.6cm}|m{5.5cm}|m{8cm}|m{2cm}|c|} \hline
            id                       & Sujet                                                                                                                                         & Test d'acceptance                                                                                                                                                                                                          & Méthode de test & Résultat \\ \hline
            515                      & TACHE Créer une fonction permettant de calculer le nombre de tatamis nécessaires étant donné des dimensions de dojo et qu'une solution existe & Quand la fonction est executée, elle retourne le nombre de tatamis nécessaires                                                                                                                                             & Automatisé      & OK       \\ \hline

            \multirow{3}{0.6cm}{514} & \multirow{3}{5.5cm}{TACHE Permettre au gestionnaire de dojo de poser cette question}                                                          & 1. Quand le programme est lancé, il demande à l'utilisateur la largeur du dojo                                                                                                                                             & Automatisé      & OK       \\ \cline{3-5}
                                     &                                                                                                                                               & 2. Quand le programme est lancé, il demande à l'utilisateur la longueur du dojo                                                                                                                                            & Automatisé      & OK       \\ \cline{3-5}
                                     &                                                                                                                                               & 3. Quand le programme est lancé, une option est disponible pour l'utilisateur pour poser cette question.                                                                                                                   & Automatisé      & OK       \\ \hline
            \multirow{2}{0.6cm}{513} & \multirow{2}{5.5cm}{US Comprendre le nombre de dispositions possibles}                                                                        & \cellcolor{tsgrey} 1. Étant donné que l'utilisateur saisie des dimensions de dojo, quand il saisie 2, il obtient le nombre conformément à l'article de Ruskey et Woodcock de 2009                                        & Automatisé      & OK       \\ \cline{3-5}
                                     &                                                                                                                                               & \cellcolor{tsgrey} 2. Étant donné que l'utilisateur saisie des dimensions de dojo en inversant la longueur et la largeur, quand il saisie 2, il obtient le nombre conformement a l'article de Ruskey et Woodcock de 2009 & Automatisé      & OK       \\ \hline

            512                      & TACHE Utiliser la fonction pour renvoyer la réponse au gestionnaire de dojo	Alpha                                                              & Quand le programme est lancé et que l'utilisateur saisie comme attendu, une réponse lui est renvoyée dans la console                                                                                                       & Automatisé      & OK       \\ \hline
            \multirow{2}{0.6cm}{511} & \multirow{2}{5.5cm}{TACHE Demander les dimensions et permettre au gestionnaire de dojo de les saisir}                                         & 1. Quand le programme est lancée, il demande a l'utilisateur la largeur du dojo                                                                                                                                            & Automatisé      & OK       \\ \cline{3-5}
                                     &                                                                                                                                               & 2. Quand le programme est lancé, il demande a l'utilisateur la longueur du dojo                                                                                                                                            & Automatisé      & OK       \\ \hline
            \multirow{2}{0.6cm}{510} & \multirow{2}{5.5cm}{US Comprendre si une solution existe}                                                                                     & \cellcolor{tsgrey} 1. Etant donné que l'utilisateur saisie des dimensions de dojo n'ayant pas de solution, quand il saisie 1, il obtient 'Il n'existe pas de disposition possible avec des tatamis 2x1 pour ce dojo'     & Automatisé      & OK       \\ \cline{3-5}
                                     &                                                                                                                                               & \cellcolor{tsgrey} 2. Étant donné que l'utilisateur saisie des dimensions de dojo ayant au moins une disposition, quand il saisie 1, il obtient 'Il existe au moins une disposition avec des tatamis 2x1 pour ce dojo'   & Automatisé      & OK       \\ \hline
        \end{testtabular}}
\end{adjustwidth}


\noindent%
\begin{adjustwidth}{-1.5cm}{0cm}

    \renewcommand{\arraystretch}{1.2}
    {\setlength{\tabcolsep}{1.5 mm}
        \begin{testtabular}{|m{0.6cm}|m{5.5cm}|m{8cm}|m{2cm}|c|} \hline
            id                                                                                                       & Sujet                                                                                                                                  & Test d'acceptance                                                                                                                                                                  & Méthode de test & Résultat \\ \hline
            \multirow{2}{0.6cm}{509}                                                                                 & \multirow{2}{5.5cm}{US Créer la fonction permettant de calculer le nombre de solutions au problème étant donné les dimensions du dojo} & 1. Quand la fonction est exécutée, elle retourne le nombre de dispositions possibles conformément a l'article de Ruskey et Woodcock de 2009                                        & Automatisé      & OK       \\ \cline{3-5}
                                                                                                                     &                                                                                                                                        & 2. Quand la fonction est exécutée en inversant la longueur et la largeur, elle retourne le nombre de dispositions possibles conformément a l'article de Ruskey et Woodcock de 2009 & Automatisé      & OK       \\ \hline

            509                                                                                                      & EPIC Développer pour un gestionnaire de dojo un outil simple lui permettant de saisir les dimensions du dojo et
            de savoir si un solution existe, le nombre de tatamis nécessaires et le nombre de dispositions possibles & \cellcolor{tsgrey} cf. tests utilisateurs des US                                                                                     &                                                                                                                                                                                    &                            \\ \hline
            \multirow{2}{0.6cm}{480}                                                                                 & \multirow{2}{5.5cm}{US Calculer le nombre de tatamis nécessaires}                                                                      & 1. Étant donne que l'utilisateur saisit des dimensions de dojo avec une solution qui existe, quand il saisie 3, il obtient le nombre de tatamis nécessaires                        & Automatisé      & OK       \\ \cline{3-5}
                                                                                                                     &                                                                                                                                        & 2. Étant donne que l'utilisateur saisit des dimensions de dojo avec aucune disposition possible, quand il saisie 3, il obtient un message indiquant l'absence de solution          & Automatisé      & OK       \\ \hline
        \end{testtabular}}
\end{adjustwidth}



\bigskip

Les tests automatisés sont définis dans le fichier \texttt{test\_alpha.py} et \texttt{test\_interface.py}. Ils peuvent être reproduits 
en l'exécutant avec ligne de commande \texttt{python3 -m pytest}.


\section{Documentation utilisateur Alpha}

\subsection{Prérequis}
%TODO faire mise en page
Configuration et installations requises:

\begin{itemize}
    \item Python 3.9 ou supérieur
    \item Librairies Python: aucune
\end{itemize}

\subsection{Comment trouver le nombre de tatamis nécessaires pour remplir un dojo?}

Lancer le programme dans un Terminal (ligne de commande: python3 interface.py)
Entrer (dans le Terminal) la longueur et la largeur du dojo en suivant les questions posées par le programme. Nb: Vous pouvez inverser la longueur et la largeur, cela n’a pas d’importance
Répondre ‘3’ à la question du programme “Que cherchez vous?”
Interpréter la réponse:
Réponse : “Le nombre de tatamis 2x1 nécessaires pour ce dojo est : [Nombre]”. Interprétation: il existe au moins une disposition possible et vous aurez besoin d’exactement [Nombre] tatamis pour remplir le dojo.
Réponse : “Il n'existe pas de disposition possible avec des tatamis 2x1 pour ce dojo”. Interprétation: il n’existe aucune disposition possible de tatamis 2 x 1 pour remplir le dojo.

\subsection{Comment savoir s’il existe une disposition possible de tatamis pour un dojo donné?}

Lancer le programme dans un Terminal (ligne de commande: python3 interface.py)
Entrer (dans le Terminal) la longueur et la largeur du dojo en suivant les questions posées par le programme. Nb: Vous pouvez inverser la longueur et la largeur, cela n’a pas d’importance
Répondre ‘1’ à la question du programme “Que cherchez vous?”
Interpréter la réponse:
Réponse : “Il existe au moins une disposition avec des tatamis 2x1 pour ce dojo”. Interprétation: il existe au moins une disposition possible.
Réponse : “Il n’existe pas de disposition possible avec des tatamis 2x1 pour ce dojo”. Interprétation: il n’existe aucune disposition possible de tatamis 2 x 1 pour remplir le dojo. Il est dans ce cas probable de devoir utiliser des demi-tatamis pour remplir pleinement le dojo.


\subsection{Comment savoir combien il existe de dispositions possibles de tatamis pour un dojo donné?}

Lancer le programme dans un Terminal (ligne de commande: python3 interface.py)
Entrer (dans le Terminal) la longueur et la largeur du dojo en suivant les questions posées par le programme. Nb: Vous pouvez inverser la longueur et la largeur, cela n’a pas d’importance
Répondre ‘2’ à la question du programme “Que cherchez vous?”
Interpréter la réponse:
a. Réponse : “Il existe [Nombre] dispositions possibles”. Interprétation: des dispositions existent pour ce dojo et il y en a [Nombre].
b. Réponse : “Il existe 0 disposition possible”. Interprétation: il n’existe pas de disposition pour ce dojo.


\section{Explication des algorithmes et choix de programmation}

\subsection{Algorithme pour les calculs de nombre de dispositions}

Nous avons tenté deux approches pour répondre à ces fonctionnalités de base:
1. Exploitation de la bibliothèque facile écrite en python pour résoudre des problèmes en programmation par contrainte. 
La publication de Xavier Olive (ref) proposant une application de cette  bibliothèque au problème qui nous concerne, 
nous avons explorer la possibilité d’adapter le programme proposé dans la publication. Si le nombre de dispositions 
proposées lors des calculs pour des dimensions de dojo données est tout à fait cohérent avec les valeurs trouvées 
dans les autres publications traitant de ce problème, il nous est rapidement apparu que les dimensions étaient limités, 
et que le temps de calcul n’était pas satisfaisant.

2. Application de l’approche proposée par Dean Hickerson (Filling rectangular rooms with Tatami mats, 2002): %TODO ajouter note bas de page
nous avons codé le programme mathématique qu’il décrit en python.

La méthode 2 a finalement été retenue car elle a l’avantage d'être très légère et rapide en exécution. 
Elle peut notamment calculer rapidement des réponses même si les dimensions sont grandes (nous n’avons d’ailleurs par 
limites les dimensions saisies a un certain nombre).


\subsection{Choix de programmation Interface}

Pour faire à l'essentiel, il a été choisi d'utiliser le terminal pour les interactions entre l'utilisateur et le programme. 
Ce n’est évidemment pas idéal, mais cela a permis de rapidement traiter la partie interface pour se concentrer sur 
les fonctionnalités et les calculs mathématiques permettant d’y répondre.

\subsection{Choix de la structure du programme}

Il nous semble important de suivre et mettre en application des principes SOLID, et en particulier du 
‘Single Responsibility Principle’ et ‘Open-Closed Principle’. 
A ce stade, vu la simplicité du programme, ce n’est pas forcément nécessaire ni applicable de manière évidente, 
mais nous souhaitons construire des fondations solides pour la suite:
Un fichier a une fonction principale
Dans cette version alpha, nous avons 2 fichiers (un fichier de front-end interface.py et un fichier de back-end alpha.py) :
interface.py : fichier contient toutes les fonctionnalités de front-end, c’est à dire l’interface utilisateur
alpha.py : fichier qui reprend les fonctions de calculs (basiques) de back-end


\section{Challenges rencontrés et apprentissage}

\subsection{Challenges rencontrés et solutions appliquées}

Les deux challenges principaux de cette version ont été les suivants:
Challenges techniques
Bien que pas très exigeante techniquement, cette version pose les fonctionnalités de base sur lesquelles 
les versions suivantes reposent. En ce qui concerne les algorithmes, le challenge principal a été la recherche 
de la documentation qui prend du temps. Mais bien que cela ait demandé du temps, nous avons eu la chance de 
trouver beaucoup de documentation de bonne qualité sur le sujet des tatamis. Il nous a ensuite été relativement 
facile d'implémenter et tester les solutions trouvées.
Challenges organisationnels
Bien que la coopération au sein de l'équipe ait débuté en phase de pré-développement, ce sprint était le premier 
qui impliquait un travail réel sur le code qui pose forcément de nouveaux challenges. Nous avons par ailleurs perdu 
définitivement un membre de l'équipe, impliquant inévitablement plus de travail de la part des membres restants. 
Pour nous permettre de bien avancer, nous avons mis en application les principes organisationnels suivants:
Sessions de travail régulières (hebdomadaires) pour discuter des points ouverts et répartir les tâches. 
Fréquence plutôt élevée pour garder un bon rythme de travail et éviter les accoups.
Retranscription écrite claire des tâches à effectuer avec les dates butoires (et dépendances de tâches lors qu’il y en a) 
avec un Compte Rendu écrit pour chaque session de travail
Communication entre les sessions de travail (par Slack), notamment pour les discussions sur l'exécution des tâches 
pré-requises à d'autres tâches dépendantes.
Travail personnel entre les session pour accomplir les tâches
Par ailleurs, pour gérer le code, github (intégré à Slack pour recevoir les notifications lors des updates) a été utilisé. 

\subsection{Apprentissage}

Les enseignements principaux de ce sprint sont les suivants:
Le choix de méthodologie agile confirmé comme étant la bonne approche pour travailler sur le projet. 
La perte d’un membre de l'équipe aurait notamment été plus difficile à gérer si tout avait été planifié de manière rigide au départ
Le bon fonctionnement de la coopération et organisation choisie (régularité des sessions de travail, documentation des sessions 
de travail avec des tâches et dates butoires claires,...)
