\section{Backlog/roadmap du sprint Beta}

L’objectif de ce sprint est de donner bien plus de valeur à l'utilisateur par:
\begin{enumerate}
    \item Une visualisation des dispositions possible (ce qui apporte bien plus à l'utilisateur que la version Alpha)
    \item Une utilisation facilité par une interface utilisateur intuitive et facile
\end{enumerate}

Pour ce faire le backlog suivant a été "embarqué" dans cette version et a résulté en la roadmap suivante:


\section{Tests}


\noindent%
\begin{adjustwidth}{-1.5cm}{0cm}

    \renewcommand{\arraystretch}{1.2}
    {\setlength{\tabcolsep}{1.5 mm}
    
        \begin{testtabular}{|m{0.6cm}|m{5.5cm}|m{8cm}|m{2cm}|c|} \hline
            id                       & Sujet                                                                   & Test d'acceptance                                                                                             & Méthode de test & Résultat \\ \hline
            \multirow{2}{0.6cm}{539} & \multirow{2}{5.5cm}{TACHE Cliquer pour avoir accès aux fonctionnalités} & Quand l'application est lancée, les boutons s'affichent                                                       & Manuel          & OK       \\ \cline{3-5}
                                     &                                                                         & Quand l'utilisateur clique sur un bouton, la fonctionnalité est activée.                                      & Manuel          & OK       \\ \hline
            \multirow{3}{0.6cm}{538} & \multirow{3}{5.5cm}{TACHE Valider les dimensions}                       & Quand l'application est lancée, seules des valeurs entières peuvent être entrées                              & Manuel          & OK       \\ \cline{3-5}
                                     &                                                                         & Quand l'application est lancée, les valeurs entrables sont restreintes aux valeurs dans un intervalle défini. & Manuel          & OK       \\ \cline{3-5}
                                     &                                                                         & Quand l'utilisateur omet ou entre 0 pour au moins une dimension, un message d'erreur s'affiche.               & Manuel          & OK       \\ \hline
        \end{testtabular}}
\end{adjustwidth}


\noindent%
\begin{adjustwidth}{-1.5cm}{0cm}

    \renewcommand{\arraystretch}{1.2}
    {\setlength{\tabcolsep}{1.5 mm}
        
            \begin{testtabular}{|m{0.6cm}|m{5.5cm}|m{8cm}|m{2cm}|c|} \hline
                id                       & Sujet                                                                                           & Test d'acceptance                                                                                                                                                & Méthode de test & Résultat \\ \hline
                537                      & TACHE Offrir la possibilité d'entrer des dimensions de dojo                                     & Quand l'application est lancée, une fenêtre s'ouvre avec la possibilité d'entrer les dimensions.                                                                 & Manuel          & OK       \\ \hline
                536                      & TACHE Créer la fonction permettant d'afficher toutes les solutions                              & Quand la fonction reçoit en input les coordonnées des tatamis pour une disposition dimensions, alors elle retourne un graph avec toutes les solutions possibles" & Manuel          & OK       \\ \hline
                535                      & TACHE Créer la fonction permettant d'afficher une seule solution                                & Quand la fonction reçoit en input les coordonnées des tatamis pour une disposition dimensions, alors elle retourne un graph avec une solution possible"          & Manuel          & OK       \\ \hline
                534                      & TACHE Créer la fonction permettant de calculer les coordonnées des tatamis pour une disposition & Quand la fonction reçoit en input les dimensions d'un dojo, alors elle retourne les coordonnées des tatamis pour une disposition"                                & Manuel          & OK       \\ \hline
                533                      & TACHE Créer la fonction permettant de choisir d'afficher toutes les solutions                   & Étant donné les inputs des dimensions d'un dojo, quand cette fonction est choisie, elle retourne un graph avec toutes les solutions possibles"                   & Manuel          & OK       \\ \hline
                532                      & US Disposer d'une interface d'affichage ergonomique                                             & Quand le programme est lancé, il ouvre une interface ergonomique"                                                                                                & Manuel          & OK       \\ \hline
                531                      & TACHE Créer une classe de fenêtre type permettant d'avoir un affichage reproductible            & Quand l'application est lancée, une fenêtre s'ouvre avec les bonnes (adaptées à l'écran) et memes dimensions"                                                    & Manuel          & OK       \\ \hline
                530                      & TACHE Créer la fonction permettant de choisir d'afficher une seule solution                     & Étant donné les inputs des dimensions d'un dojo, quand cette fonction est choisie, elle retourne un graph avec une seule solution possible"                      & Manuel          & OK       \\ \hline
                \multirow{2}{0.6cm}{519} & \multirow{2}{5.5cm}{EPIC Permettre au gestionnaire de dojo de visualiser les solutions}         & cf. tests utilisateurs des US                                                                                                                                    &                 & OK       \\ \cline{3-5}
                                         &                                                                                                 & Le nombre de solution du programme donne le meme nombre de solution que le calcul de coordonnées Tatamis                                                         & Automatisé      & OK       \\ \hline
                478                      & US Afficher visuellement toutes les dispositions possibles                                      & Étant donné des dimensions d'un dojo saisies, quand il sélectionne cette option, il obtient visuellement toutes les dispositions possibles"                      & Manuel          & OK       \\ \hline
                476                      & US Afficher visuellement une disposition possible                                               & Étant donné des dimensions d'un dojo saisies, quand il sélectionne cette option, il obtient visuellement une disposition possible"                               & Manuel          & OK       \\ \hline
            \end{testtabular}
    }
\end{adjustwidth}

% Nb: Tests automatisés réalisés avec le fichier “test_beta.py”. Ils peuvent être reproduits en l'exécutant avec ligne de commande “python3 -m pytest”.


\section{Documentation utilisateur Beta}

\subsection{Comment trouver le nombre de tatamis nécessaires?}

\begin{enumerate}
    \item Lancer l’interface (dans le Terminal avec la ligne de commande: python3 interface.py)
    \item Entrer dans l’interface la longueur et la largeur du dojo dans les champs prévus à cet effet.
          \emph{Nb: Vous pouvez inverser la longueur et la largeur, cela n’a pas d’importance}
    \item Cliquer sur le bouton: “Savoir si il existe une disposition pour le dojo”
          \emph{ Nb: Si vous oubliez d’entrer une dimension ou si vous entrez une dimension 0,
              un message d’erreur “Erreur de saisie des dimensions. Aucune dimension ne peut avoir une valeur nulle ou vide” apparaît.}
    \item Interpréter la réponse:
          \begin{enumerate}
              \item  Réponse: “Le nombre de tatamis 2x1 nécessaires pour ce dojo est :
                    [Nombre]”. Interprétation: il existe au moins une disposition possible et vous aurez besoin d’exactement [Nombre] tatamis pour remplir le dojo.
              \item  Réponse: “Le nombre de tatamis 2x1 nécessaires pour ce dojo est :
                    Aucune disposition possible de tatamis 2x1 pour ce dojo”. Interprétation:
                    il n’existe aucune disposition possible de tatamis 2 x 1 pour remplir le dojo.
          \end{enumerate}
\end{enumerate}

\subsection{Comment savoir s’il existe une disposition possible de tatamis pour un dojo donné?}

\begin{enumerate}
    \item Lancer l’interface (dans le Terminal avec la ligne de commande: python3 interface.py)
    \item Entrer dans l’interface la longueur et la largeur du dojo dans les champs prévus à cet effet.
          \emph{Nb: Vous pouvez inverser la longueur et la largeur, cela n’a pas d’importance}.
    \item Cliquer sur le bouton: “Connaître le nombre de dispositions possibles” \emph{Nb: Si vous oubliez d’entrer une dimension ou
              si vous entrez une dimension 0, un message d’erreur “Erreur de saisie des dimensions.
              Aucune dimension ne peut avoir une valeur nulle ou vide” apparaît.}
    \item Interpréter la réponse:
          \begin{enumerate}
              \item     Réponse: “Il existe au moins une disposition avec des tatamis 2x1 pour ce dojo”.
                    Interprétation: il existe au moins une disposition possible.
              \item     Réponse: “Il n’existe pas de disposition possible avec des tatamis 2x1 pour ce dojo”.
                    Interprétation: il n’existe aucune disposition possible de tatamis 2 x 1 pour remplir le dojo.
                    Il est dans ce cas probable de devoir utiliser des demi-tatamis pour remplir pleinement le dojo.
          \end{enumerate}

\end{enumerate}

\subsection{Comment savoir combien il existe de dispositions possibles de tatamis pour un dojo donné?}

\begin{enumerate}
    \item Lancer l’interface (dans le Terminal avec la ligne de commande: python3 interface.py)
    \item Entrer dans l’interface la longueur et la largeur du dojo dans les champs prévus à cet effet. Nb: Vous pouvez inverser la longueur et la largeur, cela n’a pas d’importance
    \item Cliquer sur le bouton: “Connaître le nombre de tatamis 2x1 nécessaires pour la taille du dojo”.
          \emph{Nb: Si vous oubliez d’entrer une dimension ou si vous entrez une dimension 0, un message d’erreur “Erreur de saisie des dimensions. Aucune dimension ne peut avoir une valeur nulle ou vide” apparaît.
          }
    \item Interpréter la réponse:
          \begin{enumerate}
              \item Réponse: “Le nombre de tatamis 2x1 nécessaires pour ce dojo est :  [Nombre] disposition possible”.
                    Interprétation: il faut [Nombre] tatamis pour remplir le dojo.
              \item Réponse: “Il n'existe pas de disposition possible avec des tatamis 2x1 pour ce dojo”.
                    Interprétation: la demande est non pertinente car il n’existe pas de disposition possible de tatamis 2x1 pour les dimensions du dojo.
          \end{enumerate}
\end{enumerate}


\subsection{Comment afficher une disposition?}

\subsection{Comment afficher toutes les dispositions possibles?}

\section{Explication des algorithmes et choix de programmation}

\subsection{Algorithme pour l’affichage des dispositions}

\subsection{Choix de programmation Interface}

Les choix importants concernant l’interface ont été les suivants:


\subsubsection{Choix de la librairie d’interface graphique}

Une recherche initiale a permis d'identifier 5 librairies à notre disposition:

\begin{itemize}
    \item PyQt5
    \item Tkinter
    \item Pyside2
    \item Kivy
    \item wxPython
\end{itemize}

Trois critères de choix ont été appliqués:

\begin{itemize}
    \item Capacités de la librairie
    \item Disponibilité de la documentation et ressources en ligne
    \item Connaissances préalables par l'équipe et simplicité
\end{itemize}

L'évaluation a conclu aux résultats suivants:\\


\noindent%
\begin{adjustwidth}{-1.5cm}{0cm}

    \renewcommand{\arraystretch}{1.2}
    {\setlength{\tabcolsep}{1.5 mm}
               

            \begin{testtabular}{*{6}{|p{0.16\linewidth} }|} \hline                

                \rowcolor{tsgrey}                               &                                                                                                & Tkinter & Pyside2 & Kivy & wxPython \\ \hline
                \cellcolor{lightgray} Capacités de la librairie & Très large variété de widgets (boutons, champs de saisie, boîtes de message…)

                Beaucoup de possibilités

                Grille disponible
                                                                & Large variété de widgets (boutons, champs de saisie, boîtes de message…)

                Grille disponible (mais également une fonctionnalité qui facilite le placement)
                                                                & Large variété de widgets (boutons, champs de saisie, boîtes de message…)

                Grille disponible
                                                                & Widgets disponibles, dont certains bien designés mais d’autres moins intuitifs (par ex. les boutons)

                Grille disponible
                                                                & Large variété de widgets (boutons, champs de saisie, boîtes de message…)

                Grille disponible 
                                                                                                                                                                                  \\ \hline
            \end{testtabular}
        
    }

\end{adjustwidth}

\section{Challenges rencontrés et apprentissage}