\section{Backlog/roadmap du sprint Production}

L’objectif de ce sprint est de finaliser en augmentant encore la valeur donné à l'utilisateur par:
Ajout de deux fonctionnalités concernant les dispositions symétriques. L’objectif étant d'éliminer ces dispositions symétriques qui apportent peu, voire prêtent à confusion pour l’utilisateur.
Dernière amélioration mineure de l’interface, surtout sur le visuel, pour aligner le design des messages de réponse à celui de l’interface principale.\\

Pour ce faire le backlog suivant a été \emph{embarqué} dans cette version et a résulté en la roadmap suivante:

\section{Tests}

Les tests réalisés pour cette version et leurs résultats sont les suivants:\\


\noindent%
\begin{adjustwidth}{-1.5cm}{0cm}

    \renewcommand{\arraystretch}{1.2}
    {\setlength{\tabcolsep}{1.5 mm}
        \begin{testtabular}{|m{0.6cm}|m{5.5cm}|m{8cm}|m{2cm}|c|} \hline
            id  & Sujet                                                                                                  & Test d'acceptance                                                                                                                                                                                        & Méthode de test & Résultat \\ \hline
            523 & US Comprendre le nombre de dispositions possibles excluant les dispositions symétriques                & Étant donné que l'utilisateur saisie des dimensions valides de dojo, il obtient le nombre de solutions possibles en excluant les solutions symétriques                                                   & Manuel          & OK       \\ \hline
            524 & US Afficher visuellement toutes les dispositions possibles excluant les dispositions symétriques       & \cellcolor{tsgrey} Étant donné des dimensions d'un dojo saisies, quand il sélectionne cette option, il obtient visuellement toutes les dispositions possibles en excluant les solutions symétriques      & Manuel          & OK       \\ \hline
        \end{testtabular}}
\end{adjustwidth}

\bigskip

Par ailleurs, étant donné qu’il s’agit de la dernière version pour mise en production,
il est important de tester l’ensemble des fonctionnalités du programme. Des tests de non régression
ont été effectués:

%TODO inclure tableaux tests non regression

\bigskip
Erreur sur validation des saisies:

Une erreur a été constaté suivant les versions utilisés de PyQt:
\begin{itemize}
    \item Version 5.9.7: la validation de la saisie des données fonctionne (comportement attendu)
    \item Version 5.12.2: la validation fonctionne partiellement (cela limite bien le type d'entrée - uniquement des entiers, et limite le nombre de chiffres, mais cela ne limite pas à 25 ou 300 comme attendu; cela limite uniquement à respectivement 99 et 999). Pour corriger ce changement de comportement de QIntValidator, un nouveau message d’erreur a été mis en place.
\end{itemize}

\section{Documentation Utilisateur Production}

\subsection{Prérequis}

Configuration et installations requises:

\begin{itemize}
    \item Python 3.9 ou supérieur
    \item Librairies Python: datetime, numpy, PyQt5.QtCore, PyQt5.QtGui, PyQt5.QtWidgets, sys, time.
          (Idéalement la version 5.9.7 de PyQt)
    \item Installer les polices Nexa (light et bold) (fichiers otf fournis)
\end{itemize}
