\section{Méthode et outil de test}

Les test automatisées sont réalisés à l'aide de la bibliothèque \textbf{pytest} . Les fonctions et les classes testées sont
appelées par des fonctions de test dédiées qui vérifie la concordance des valeurs retournées avec ce qui est attendu.



\section{Version Alpha}


\noindent%
\begin{adjustwidth}{-1.5cm}{0cm}

    \renewcommand{\arraystretch}{1.2}
    {\setlength{\tabcolsep}{1.5 mm}
        \begin{tabular}{|m{0.6cm}|m{5.5cm}|m{8cm}|m{2cm}|c|} \hline
            id                                                                             & Sujet                                                                                & Test d'acceptance                                                                                        & Méthode de test & Résultat \\ \hline
            518                                                                            & TACHE Utiliser la fonction pour renvoyer la réponse au gestionnaire de dojo          &
            Quand le programme est lancé et que l'utilisateur saisi comme attendu,
            une réponse lui est renvoyée dans la console                                   & Manuel                                                                               & OK                                                                                                                                    \\ \hline
            517                                                                            & TACHE Créer une fonction permettant de calculer le nombre de tatamis nécessaires
            étant donné des dimensions de dojo et qu'une solution existe                   &
            Quand la fonction est exécutée, elle retourne le nombre de tatamis nécessaires &
            Automatisé                                                                     & OK                                                                                                                                                                                                                           \\ \hline
            \multirow{3}{0.6cm}{516}                                                       & \multirow{3}{5.5cm}{TACHE Permettre au gestionnaire de dojo de poser cette question} & 1. Quand le programme est lancé, il demande a l'utilisateur la largeur du dojo                           & Automatisé      & OK       \\ \cline{3-5}
                                                                                           &                                                                                      & 2. Quand le programme est lancé, il demande a l'utilisateur la longueur du dojo                          & Automatisé      & OK       \\ \cline{3-5}
                                                                                           &                                                                                      & 3. Quand le programme est lancé, une option est disponible pour l'utilisateur pour poser cette question. & Automatisé      & OK       \\ \hline
        \end{tabular}}
\end{adjustwidth}


\noindent%
\begin{adjustwidth}{-1.5cm}{0cm}

    \renewcommand{\arraystretch}{1.2}
    {\setlength{\tabcolsep}{1.5 mm}
        \begin{tabular}{|m{0.6cm}|m{5.5cm}|m{8cm}|m{2cm}|c|} \hline
            id                       & Sujet                                                                                                                                         & Test d'acceptance                                                                                                                                                                                                          & Méthode de test & Résultat \\ \hline
            515                      & TACHE Créer une fonction permettant de calculer le nombre de tatamis nécessaires étant donné des dimensions de dojo et qu'une solution existe & Quand la fonction est executée, elle retourne le nombre de tatamis nécessaires                                                                                                                                             & Automatisé      & OK       \\ \hline

            \multirow{3}{0.6cm}{514} & \multirow{3}{5.5cm}{TACHE Permettre au gestionnaire de dojo de poser cette question}                                                          & 1. Quand le programme est lancé, il demande à l'utilisateur la largeur du dojo                                                                                                                                             & Automatisé      & OK       \\ \cline{3-5}
                                     &                                                                                                                                               & 2. Quand le programme est lancé, il demande à l'utilisateur la longueur du dojo                                                                                                                                            & Automatisé      & OK       \\ \cline{3-5}
                                     &                                                                                                                                               & 3. Quand le programme est lancé, une option est disponible pour l'utilisateur pour poser cette question.                                                                                                                   & Automatisé      & OK       \\ \hline
            \multirow{2}{0.6cm}{513} & \multirow{2}{5.5cm}{US Comprendre le nombre de dispositions possibles}                                                                        & \cellcolor{tsgrey} 1. Étant donné que l'utilisateur saisie des dimensions de dojo, quand il saisie 2, il obtient le nombre conformément à l'article de Ruskey et Woodcock de 2009                                        & Automatisé      & OK       \\ \cline{3-5}
                                     &                                                                                                                                               & \cellcolor{tsgrey} 2. Étant donné que l'utilisateur saisie des dimensions de dojo en inversant la longueur et la largeur, quand il saisie 2, il obtient le nombre conformement a l'article de Ruskey et Woodcock de 2009 & Automatisé      & OK       \\ \hline

            512                      & TACHE Utiliser la fonction pour renvoyer la réponse au gestionnaire de dojo	Alpha                                                              & Quand le programme est lancé et que l'utilisateur saisie comme attendu, une réponse lui est renvoyée dans la console                                                                                                       & Automatisé      & OK       \\ \hline
            \multirow{2}{0.6cm}{511} & \multirow{2}{5.5cm}{TACHE Demander les dimensions et permettre au gestionnaire de dojo de les saisir}                                         & 1. Quand le programme est lancée, il demande a l'utilisateur la largeur du dojo                                                                                                                                            & Automatisé      & OK       \\ \cline{3-5}
                                     &                                                                                                                                               & 2. Quand le programme est lancé, il demande a l'utilisateur la longueur du dojo                                                                                                                                            & Automatisé      & OK       \\ \hline
            \multirow{2}{0.6cm}{510} & \multirow{2}{5.5cm}{US Comprendre si une solution existe}                                                                                     & \cellcolor{tsgrey} 1. Etant donné que l'utilisateur saisie des dimensions de dojo n'ayant pas de solution, quand il saisie 1, il obtient 'Il n'existe pas de disposition possible avec des tatamis 2x1 pour ce dojo'     & Automatisé      & OK       \\ \cline{3-5}
                                     &                                                                                                                                               & \cellcolor{tsgrey} 2. Étant donné que l'utilisateur saisie des dimensions de dojo ayant au moins une disposition, quand il saisie 1, il obtient 'Il existe au moins une disposition avec des tatamis 2x1 pour ce dojo'   & Automatisé      & OK       \\ \hline
        \end{tabular}}
\end{adjustwidth}


\noindent%
\begin{adjustwidth}{-1.5cm}{0cm}

    \renewcommand{\arraystretch}{1.2}
    {\setlength{\tabcolsep}{1.5 mm}
        \begin{tabular}{|m{0.6cm}|m{5.5cm}|m{8cm}|m{2cm}|c|} \hline
            id                                                                                                       & Sujet                                                                                                                                  & Test d'acceptance                                                                                                                                                                  & Méthode de test & Résultat \\ \hline
            \multirow{2}{0.6cm}{509}                                                                                 & \multirow{2}{5.5cm}{US Créer la fonction permettant de calculer le nombre de solutions au problème étant donné les dimensions du dojo} & 1. Quand la fonction est exécutée, elle retourne le nombre de dispositions possibles conformément a l'article de Ruskey et Woodcock de 2009                                        & Automatisé      & OK       \\ \cline{3-5}
                                                                                                                     &                                                                                                                                        & 2. Quand la fonction est exécutée en inversant la longueur et la largeur, elle retourne le nombre de dispositions possibles conformément a l'article de Ruskey et Woodcock de 2009 & Automatisé      & OK       \\ \hline

            509                                                                                                      & EPIC Développer pour un gestionnaire de dojo un outil simple lui permettant de saisir les dimensions du dojo et
            de savoir si un solution existe, le nombre de tatamis nécessaires et le nombre de dispositions possibles & \cellcolor{tsgrey} cf. tests utilisateurs des US                                                                                     &                                                                                                                                                                                    &                            \\ \hline
            \multirow{2}{0.6cm}{480}                                                                                 & \multirow{2}{5.5cm}{US Calculer le nombre de tatamis nécessaires}                                                                      & 1. Étant donne que l'utilisateur saisit des dimensions de dojo avec une solution qui existe, quand il saisie 3, il obtient le nombre de tatamis nécessaires                        & Automatisé      & OK       \\ \cline{3-5}
                                                                                                                     &                                                                                                                                        & 2. Étant donne que l'utilisateur saisit des dimensions de dojo avec aucune disposition possible, quand il saisie 3, il obtient un message indiquant l'absence de solution          & Automatisé      & OK       \\ \hline
        \end{tabular}}
\end{adjustwidth}


\section{Version Beta}

\noindent%
\begin{adjustwidth}{-1.5cm}{0cm}

    \renewcommand{\arraystretch}{1.2}
    {\setlength{\tabcolsep}{1.5 mm}
        \begin{tabular}{|m{0.6cm}|m{5.5cm}|m{8cm}|m{2cm}|c|} \hline
            id                       & Sujet                                                                   & Test d'acceptance                                                                                             & Méthode de test & Résultat \\ \hline
            \multirow{2}{0.6cm}{539} & \multirow{2}{5.5cm}{TACHE Cliquer pour avoir accès aux fonctionnalités} & Quand l'application est lancée, les boutons s'affichent                                                       & Manuel          & OK       \\ \cline{3-5}
                                     &                                                                         & Quand l'utilisateur clique sur un bouton, la fonctionnalité est activée.                                      & Manuel          & OK       \\ \hline
            \multirow{3}{0.6cm}{538} & \multirow{3}{5.5cm}{TACHE Valider les dimensions}                       & Quand l'application est lancée, seules des valeurs entières peuvent être entrées                              & Manuel          & OK       \\ \cline{3-5}
                                     &                                                                         & Quand l'application est lancée, les valeurs entrables sont restreintes aux valeurs dans un intervalle défini. & Manuel          & OK       \\ \cline{3-5}
                                     &                                                                         & Quand l'utilisateur omet ou entre 0 pour au moins une dimension, un message d'erreur s'affiche.               & Manuel          & OK       \\ \hline
        \end{tabular}}
\end{adjustwidth}


\noindent%
\begin{adjustwidth}{-1.5cm}{0cm}

    \renewcommand{\arraystretch}{1.2}
    {\setlength{\tabcolsep}{1.5 mm}
        \begin{tabular}{|m{0.6cm}|m{5.5cm}|m{8cm}|m{2cm}|c|} \hline
            id                       & Sujet                                                                                           & Test d'acceptance                                                                                                                                                & Méthode de test & Résultat \\ \hline
            537                      & TACHE Offrir la possibilité d'entrer des dimensions de dojo                                     & Quand l'application est lancée, une fenêtre s'ouvre avec la possibilité d'entrer les dimensions.                                                                 & Manuel          & OK       \\ \hline
            536                      & TACHE Créer la fonction permettant d'afficher toutes les solutions                              & Quand la fonction reçoit en input les coordonnées des tatamis pour une disposition dimensions, alors elle retourne un graph avec toutes les solutions possibles" & Manuel          & OK       \\ \hline
            535                      & TACHE Créer la fonction permettant d'afficher une seule solution                                & Quand la fonction reçoit en input les coordonnées des tatamis pour une disposition dimensions, alors elle retourne un graph avec une solution possible"          & Manuel          & OK       \\ \hline
            534                      & TACHE Créer la fonction permettant de calculer les coordonnées des tatamis pour une disposition & Quand la fonction reçoit en input les dimensions d'un dojo, alors elle retourne les coordonnées des tatamis pour une disposition"                                & Manuel          & OK       \\ \hline
            533                      & TACHE Créer la fonction permettant de choisir d'afficher toutes les solutions                   & Étant donné les inputs des dimensions d'un dojo, quand cette fonction est choisie, elle retourne un graph avec toutes les solutions possibles"                   & Manuel          & OK       \\ \hline
            532                      & US Disposer d'une interface d'affichage ergonomique                                             & Quand le programme est lancé, il ouvre une interface ergonomique"                                                                                                & Manuel          & OK       \\ \hline
            531                      & TACHE Créer une classe de fenêtre type permettant d'avoir un affichage reproductible            & Quand l'application est lancée, une fenêtre s'ouvre avec les bonnes (adaptées à l'écran) et memes dimensions"                                                    & Manuel          & OK       \\ \hline
            530                      & TACHE Créer la fonction permettant de choisir d'afficher une seule solution                     & Étant donné les inputs des dimensions d'un dojo, quand cette fonction est choisie, elle retourne un graph avec une seule solution possible"                      & Manuel          & OK       \\ \hline
            \multirow{2}{0.6cm}{519} & \multirow{2}{5.5cm}{EPIC Permettre au gestionnaire de dojo de visualiser les solutions}         & cf. tests utilisateurs des US                                                                                                                                    &                 & OK       \\ \cline{3-5}
                                     &                                                                                                 & Le nombre de solution du programme donne le meme nombre de solution que le calcul de coordonnées Tatamis                                                         & Automatisé      & OK       \\ \hline
            478                      & US Afficher visuellement toutes les dispositions possibles                                      & Étant donné des dimensions d'un dojo saisies, quand il sélectionne cette option, il obtient visuellement toutes les dispositions possibles"                      & Manuel          & OK       \\ \hline
            476                      & US Afficher visuellement une disposition possible                                               & Étant donné des dimensions d'un dojo saisies, quand il sélectionne cette option, il obtient visuellement une disposition possible"                               & Manuel          & OK       \\ \hline
        \end{tabular}}
\end{adjustwidth}



\section{Version Release}

\noindent%
\begin{adjustwidth}{-1.5cm}{0cm}

    \renewcommand{\arraystretch}{1.2}
    {\setlength{\tabcolsep}{1.5 mm}
        \begin{tabular}{|m{0.6cm}|m{5.5cm}|m{8cm}|m{2cm}|c|} \hline
            id  & Sujet                                                                                                              & Test d'acceptance                                                                                                                                                                                                                  & Méthode de test & Résultat \\ \hline
            540 & EPIC Amélioration de l'expérience par ajout de fonctionnalités et amélioration de l'interface                      & \cellcolor{tsgrey}cf tests utilisateur des US                                                                                                                                                                                                        & Manuel          & OK       \\ \hline
            481 & US Affichage des dimensions sur les représentation graphiques des dispositions de tatamis                          & L'affichage graphique des solutions comprend un rappel des dimensions du dojo.                                                                                                                                                     & Manuel          & OK       \\ \hline
            482 & US Modification des dimensions du dojo                                                                             & L'application permet de saisir d'autres dimensions que celles initialement renseignées.                                                                                                                                            & Manuel          & OK       \\ \hline
            482 & US Modification des dimensions du dojo                                                                             & L'application permet de saisir d'autres dimensions que celles initialement renseignées.                                                                                                                                            & Manuel          & OK       \\ \hline
            484 & US Comprendre s'il est utile d'utiliser des demi-tatamis pour trouver une solution                                 & Si il n'existe pas de solutions avec des tatamis entiers, l'application affiche qu'il est toujours possible d'obtenir une solutions avec des demi-tatamis.                                                                         & Manuel          & OK       \\ \hline
            525 & US Obtenir une solution étant donné un nombre de tatamis                                                           & Lorsque l'utilisateur entre un nombre de tatamis, l'application propose une solution dont une des dimensions ne peut-être inférieure à 3 et dont le rapport largeur/longueur ne peut être inférieur à 1/3.& Manuel          & OK       \\ \hline
            551 & TACHE créer une fonction pour obtenir une solution étant donné un nombre de tatamis                                & Lorsque l'utilisateur entre un nombre de tatamis, l'application propose une solution dont une des dimensions ne peut-être inférieure à 3 et dont le rapport largeur/longueur ne peut être inférieur à 1/3.                         & Manuel          & OK       \\ \hline
            552 & TACHE permettre à l'utilisateur d'utiliser la fonction pour obtenir une solution étant donné un nombre de tatamis  & \cellcolor{tsgrey}La fonctionnalité est disponible sur l'interface utilisateur                                                                                                                                                                       & Manuel          & OK       \\ \hline
            527 & US calculer les dispositions pour des dimensions plus petites                                                      & \cellcolor{tsgrey} Lorsque l'utilisateur propose des dimensions, il n'existe pas de disposition aux dimensions plus grandes entre la(les) solution(s) alternative(s) proposée(s) par l'application et la demande de l'utilisateur. & Manuel          & OK       \\ \hline
            544 & TACHE créer une fonction pour calculer les dispositions pour des dimensions plus petites                           & Lorsque l'utilisateur propose des dimensions, il n'existe pas de disposition aux dimensions plus grandes entre la(les) solution(s) alternative(s) proposée(s) par l'application et la demande de l'utilisateur.                    & Automatisé      & OK       \\ \hline
            545 & TACHE permettre à l'utilisateur d'utiliser la fonction de calcul des dispositions pour des dimensions plus petites & \cellcolor{tsgrey}La fonctionnalité est disponible sur l'interface utilisateur                                                                                                                                                                       & Manuel          & OK       \\ \hline
                   \end{tabular}}
\end{adjustwidth}

\noindent%
\begin{adjustwidth}{-1.5cm}{0cm}

    \renewcommand{\arraystretch}{1.2}
    {\setlength{\tabcolsep}{1.5 mm}
        \begin{tabular}{|m{0.6cm}|m{5.5cm}|m{8cm}|m{2cm}|c|} \hline
            id  & Sujet                                                                                                              & Test d'acceptance                                                                                                                                                                                                                  & Méthode de test & Résultat \\ \hline
            541 & US Amélioration de l'expérience par ajout de fonctionnalités et amélioration de l'interface                        & \cellcolor{tsgrey}Chaque interaction proposée par l'interface correspond à une fonctionnalité bien définie et identifiable, l'interface est intuitive est agréable visuellement.                                                                     & Manuel          & OK       \\ \hline
            547 & TACHE Interface: changement de couleur de fonds                                                                    & \cellcolor{tsgrey}Test utilisateur: Le fonds de l'interface est coloré& Manuel          & OK       \\ \hline
            548 & TACHE Interface: amélioration des champs de saisie                                                                 & \cellcolor{tsgrey}Les champs de saisie sont bien identifiables& Manuel          & OK       \\ \hline
            549 & TACHE Interface: changement des couleurs et police du texte                                                        & \cellcolor{tsgrey} Le texte est coloré et avec une police lisible et moderne& Manuel          & OK       \\ \hline
            550 & TACHE Interface: redisposition et catégorisation des fonctionnalités                                               & \cellcolor{tsgrey}Les fonctionnalités sont groupées par theme sur l'interface"& Manuel          & OK       \\ \hline
            542 & US Afficher la surface                                                                                             & \cellcolor{tsgrey} La valeur d'aire affichée sur l'interface correspond à l'aire de la surface occupée par les tatamis.& Manuel          & OK       \\ \hline
        \end{tabular}}
\end{adjustwidth}